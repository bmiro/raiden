\section{Càlculs}
L'apartat de calculs inclou, per una banda, els requerits per l'encunicat a l'haver de transformar
el punt P del sistema de coordenades de la camera al del braç robot. En aquest punt es mostra
el procés de obtenció de les equacions per fer tal tranformació i en l'apartat de codi
es veu la seva implementació en el programa.

Per altra banda es mostra el calcul dels punts emprats de la situació dels objectes en l'entorn
a partir dels seus homolegs. Com s'ha mencionat la posició dels pals es relativa a un punt,
concretament tots els pals on es deixen les peces venen en funció de la posició del primer.
De la mateixa manera la posició de cada pila on inicialment es recullen les peces
ve en funció de la posició de la primera pila, el nombre de piles i la distància entre elles.
Finalment també es descriu el calcul de punts en Z on s'agafen les peces en funció del pla
de terra, l'altura de les peces i la quantitat de les mateixes.

\subsection{Transformació de sistemes de coordenades}
Amb l'enunciat tenim la posició de la càmera \{C\} respecte el robot \{R\} i
la posició del punt P, que es el centre de sistema de coordenades del palé \{P\},
respecte la càmera. El que volem es posar el punt P en el sistemes de cooerdenades del robot.

Així doncs tenim:

% \begin{description}[align=left]
% %  \item []
% \end{description}



\subsection{Calcul de piles, pals (X,Y) i altura de peces (Z)}