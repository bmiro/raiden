\section{Càlculs}
L'apartat de càlculs inclou, per una banda, els requerits per l'enunciat a
l'haver de transformar el punt P del sistema de coordenades de la càmera al del
braç robot. En aquest punt es mostra el procés de obtenció de les equacions per
fer tal transformació i en l'apartat de codi es veu la seva implementació en el
programa.

Per altra banda es mostra el càlcul dels punts emprats de la situació dels
objectes en l'entorn a partir dels seus homòlegs. Com s'ha mencionat la posició
dels pals es relativa a un punt, concretament tots els pals on es deixen les
peces venen en funció de la posició del primer.
De la mateixa manera la posició de cada pila on inicialment es recullen les
peces ve en funció de la posició de la primera pila, el nombre de piles i la
distància entre elles.

Finalment també es descriu el càlcul de punts en Z on s'agafen les peces en
funció del pla de terra, l'altura de les peces i la quantitat de les mateixes.

\subsection{Transformació de sistemes de coordenades}
Amb l'enunciat tenim la posició de la càmera \{C\} respecte el robot \{R\} i
la posició del punt P, que es el centre de sistema de coordenades del palé
\{P\}, respecte la càmera. El que volem es posar el punt P en el sistemes de
coordenades del robot.

Així doncs definim:

\begin{description}
\item[$^RT_C$] La matriu de transformació de un punt del robot a la càmera
\item[$^CT_P$] La matriu de transformació de un punt de la càmera al palé
\item[$^PP$] Punt en el sistema de coordenades del palé
\item[$^RP$] Punt en el sistema de coordenades del robot
\end{description}

Combinant aquestes matrius obtenim que:

$$ ^RT_C \times \,^CT_P \times \,^PP = \,^RP $$

Per tant hem de cercar $^RT_P$:

$$ ^RT_P = \,^RT_C \times \,^CT_P $$

$$ ^RT_P = 
\left( \begin{array}{ccc|c}
0 & 1 &  0 & cPx\\
1 & 0 &  0 & cPy\\
0 & 0 & -1 &   0\\
\hline
0 & 0 &  0 & 1
\end{array} \right) \times
\left( \begin{array}{ccc|c}
cos \phi & -sin \phi &  0 & oPx\\
sin \phi &  cos \phi &  0 & oPy\\
0        &         0 &  1 &   0\\
\hline
0 & 0 &  0 & 1
\end{array} \right) $$

$$ ^RT_P = 
\left( \begin{array}{ccc|c}
sin \phi &  cos \phi &  0 & oPy + cPx\\
cos \phi & -sin \phi &  0 & oPx + cPy\\
0        &         0 & -1 &   0\\
\hline
0 & 0 &  0 & 1
\end{array} \right)$$

Així doncs per passar d'un punt del sistema de coordenades $^PP$ a $^RP$ ho
feim així:

$$^RT \times \,^PP = \,^RP$$

En el cas genèric agafam $^PP = 
\left( \begin{array}{c}
x \\
y \\
z \\
\hline
1
\end{array} \right)$

$$
\left( \begin{array}{ccc|c}
sin \phi &  cos \phi &  0 & oPy + cPx\\
cos \phi & -sin \phi &  0 & oPx + cPy\\
0        &         0 & -1 &   0\\
\hline
0 & 0 &  0 & 1
\end{array} \right)
\times
\left( \begin{array}{c}
x \\
y \\
z \\
\hline
1
\end{array} \right)
=
\left( \begin{array}{c}
x\times sin \phi + y \times cos \phi + cPy + cPx \\
x\times cos \phi - y \times sin \phi + oPx + cPy \\
-z \\
\hline
1
\end{array} \right)
$$

De on extreim les següents equacions que seran implementades dins del
nostre programa (secció \ref{cam2rob}).

$$^RP_X = x\times sin \phi + y \times cos \phi + cPy + cPx$$
$$^RP_Y = x\times cos \phi - y \times sin \phi + oPx + cPy$$
$$^RP_Z = -z$$

\subsection{Càlcul de piles, pals (X,Y) i altura de peces (Z)} \label{calcpts}
A banda de la transformació de punts entre sistemes de coordenades hi ha altres
punts a calcular.
El primer són les piles d'on s'agafen les peces. Dins del llistat de punts
tenim el punt de la primera pila, les següents es troben situades a un
desplaçament en \emph{X}. Així doncs en el nostre escenari senzillament s'ha de
incrementar en \emph{X} el valor de la distància entre piles. Succeeix exactament
el mateix amb els pals on es deixen les peces.

Els detalls de implementació i valors de les constants de separació es poden
veure en l'apartat especific \ref{codcalcpts} on s'explica el codi font. Cal
remarcar que al codi també figura un desplaçament en Y, per si es volguessin
posar les piles o pals fent una diagonal, però per defecte la \emph{macro}
està inicialitzada a 0.




