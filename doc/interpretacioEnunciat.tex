\section{Interpretació de l'enunciat}
En aquest apartat s'expliquen les extencions i suposicions de l'enunciat original.
Tal i com es demana el programa esta parametrizat segons la posició de la camera,
el punt P que aquesta detecte i l'angle $\alpha$. A més existeixen altres paràmetres
com els punts on es troben les peces originalment i on es deixen al final, aquests
s'explicaran a l'apartat de punts.

Per altra banda s'ha introduït la possibilitat de fixar un nombre diferent de numero
total de peces aixi com el seu repartiment en les piles, així doncs es poden tenir 
piles amb diferent número de peces i la possibilitat de tenir més de una peça
d'algun color o cap.

Tots aquests paràmetres es tornen a veure detallats en l'apartat d'explicació del
codi TODO LABEL.

Finalment l'enunciat deixava oberta la possibilitat de que fer amb les peces de
tipus 4, on s'ha optat per posar-les al pal 4 per aprofitar el codi ja escrit i
així seguir la coherencia i estructura dels tipus de peça anteriors. El fet de
no optar per apilar-les en qualsevol punt de l'entorn es perquè en la paletització
ja s'ha demostrat coneixement de com apliar peces i tractar el tipus 4 de manera
diferent als anteriors minvava elagància al codi i l'execució.

\subsection{Moviment del robot}
Així doncs en primer lloc el robot agafa les peces dels munts inicials, munt a munt.
Aquestes son co\lgem ocades a la zona de paletització, un cop acabat amb la pinça
tombada agafa les peces començant per les que es troben mes a la seva esquerra
per co\lgem ocarles als pals. Així doncs al diagrama seguent veim quins serien
els ordres de recollida depenent dels diferents angles:

FIGURA ORDRE RECOLLIDA

Així doncs l'ordre de despaletització depen de l'angle i no del tipus de peça.