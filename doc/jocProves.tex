\section{Joc de proves}

Per tal de provar la pràctica s'han realitzat una sèrie de jocs de proves
descrits a continuació.

Cal esmentar que al no poder oferir una traça de l'execució
aquest apartat sols serveix per mencionar proves interessants a fer
per validar qualsevol canvi el programa.

La primera prova es realitza amb l'enunciat per defecte, sols dos munts i
quatre peces, una de cada tipus. A continuació es duen a terme rotacions
del palé.

Cada una de les rotacions es fa per comprovar que el braç agafi les
peces amb l'ordre que toca sense co\lgem isonar amb els demés munts.
Així doncs es realitzen 8 execucions corresponents als 8 casos
\ref{figrecpec} de recollida que s'han esmentat a l'apartat
d'explicació de l'enunciat.

A continuació es realitza la prova de posar les 4 peces del mateix tipus,
aquest test serveix per comprovar si funciona correctament el desplaçament
en \emph{Z}. Per altra banda amb aquest test es comprova que passa
quan no existeix algun tipus de peça en la seqüència.

Finalment es prova l'execució amb piles de diferent dimensió (4 i 2)
amb peces repetides del mateix color.

Cal esmentar que tots els tests foren passats abans de l'escotament
de les bateries que guardaven els valors dels codificadors angulars
òptics. Després del re-establiment del robot sols s'han fet proves
amb les piles de diferent dimensió amb $\alpha = 133$ i $\alpha = 240$.