\section{Estructura del programa}
El programa es divideix en cinc seccions.

\begin{description}
\item [Capçalera de l'enunciat] Interficie de macros requerida per l'enunciat
on hi ha paràmetres com l'angle $\alpha$.
\item [Macros pròpies] Reassingació de valors per les \emph{features} extra que
s'han implementat com ara emprar
piles amb nombre de peces diferets, si es vol fer una execució amb l'enunciat
original sols fa falta
assignar els valors de les variables de la secció anterior.
\item [Declaracions pròpies] Aquí es declaren les variables globals propies
del programa, en principi sols les 
ha de tocar el programador.
\item [Rutines] Codi de les rutines on es desenvolupa tota la pràctica.
\item [Main] Simple crida ordenada a les rutines per desencadenar l'execució.
\end{description}

\subsection{Macros, paràmetres}
Les \emph{macros} de la \emph{capçalera de l'enunciat} ja estan explicades al
propi enunciat, aquí s'expliquen les
altres \emph{macros} introduïdes per les \emph{features} pròpies, corresponents
a l'apartat de \emph{Macros pròpies}.

Cal mencionar aquests valors no són realment \emph{macros} sinó variables, una
\emph{macro} es resol en temps de pre-processador no en temps de compilació o execució,
aquest fet porta alguns problemes mencionats a l'apartat de incidents TODOLABEL.

\begin{description}\label{macprop}
\item [PINCA] Valor de la pinça emprada, en el nostre cas el robot només en té
una i és la 1.

\item [DPALE] Retard de 4 segons requerit a l'enunciat un cop s'ha muntat el
palé.

\item [D\#PINCA\%] Retards aplicats després de O\#brir o T\#ancar la pinça en
funció de si es I\%nicial o F\%inal, abans o després d'agafar la peça.

\item [VLENT] V\_elocitat a la qual es duen a terme les aproximacions
delicades, com l'agafada de peces.

\item [VNORMAL] V\_elocitat a la que es desplaça normalment el braç en els
trajectes segurs.

\item [ZS] Pla Z on es consideren segurs els moviments del braç robot i on
es duen a terme els desplaçaments llargs.

\item [ZTPT] Pla de T\_erra (Z = 0) amb la P\_inça T\_ombada (angle inferor
a 45 amb el pla XY)

\item [ZTPR] Pla de terra amb la pinça perpendicular al pla XY  
                                                                                                         
\item [TP] T\_ipus de P\_eça diferents que tenim, es igual el numero de munts
del palé i de pals de destí.

\item [PILES] Número de piles d'on es fa la recollida inicial de peces.

\item [PALS] Número de pals de destí on es deixen les peces.

\item [DT] Retard per l'obertura i tancament de la pinça.

\item [HDISC] Altura de les peces (discs)

\item [D\#PILES] Distància en X i Y entre les piles

\item [D\#PALS] Distància en X i Y entre els pals

\item [DZPAL] Distància Z a baixar per introduir la peça dins el pal.

\item [RELPALE(T,E)] Distància relativa de la peça tipus T al punt P, en funció
de l'eix E (1 per X, 2 per Y)
\end{description}

\subsection{Estructura del codi}
En el codi s'ha fet una funció per cada acció i alguna d'auxiliar transversal.
La intenció es dedicar tota la feina a les rutines i que la lectura del
programa es pugui llegir de manera natural en els nivells més alts entrant en
detalls a mesura que es va aprofundint en les funcions.


\begin{center}
\begin{tabular}{|l|l|l|l|}
\hline
\multicolumn{2}{|c|}{Munta palé} & \multicolumn{2}{|c|}{Desmunta palé}\\
\hline
Agafa pila & Posa palé & Agafa Palé & Posa pal\\
\hline
\multicolumn{4}{|c|}{Obre/tanca pinça}\\
\hline
\end{tabular}
\end{center}


Cal fer especial menció a dos vectors de control de l'estat del programa,
aquests són \texttt{npcspila} i \texttt{alloc}.

\texttt{npcspila(p)} conté el nombre de peces que resten a la pila inicial de
on s'agafen les peces.

\texttt{alloc(m)} indica quantes peces ja estan a lloc, situades al munt
\texttt{p} del palé, recordem que cada munt correspon a un tipus de peça.

Ambdues variables serveixen per controlar l'altura en Z d'on agafar les
peces, i per sabre quan ja no en queden més.

Amb aquest principi ens queda un cos principal tan lleuger com aquest.
\begin{minted}[frame=lines, fontsize=\small]{vbnet}
960 gosub *INIT
970 gosub *CALCPTS
980 gosub *MNTPALE
990 dly DPALE
1000 gosub *DESPALE
1010 gosub *ACABA
1020 end
\end{minted}

\subsubsection{Auxiliars}
\paragraph{Canvi sistemes de coordenades}\label{cam2rob}
En la funció \texttt{CAM2ROB} es troba implementat el canvi de sistemes
de coordenades del la càmera al robot.
Com a interfície llegeix els punts \texttt{ppx} i \texttt{ppy} que són els que es desitgen
transformar, així com les \emph{macros} del programa \texttt{oPx}, \texttt{oPy},
\texttt{cPx}, \texttt{cPy} i l'\texttt{angle}. Deixa els valors de
\texttt{ppx} i \texttt{ppy} en funció del sistema de coordenades del braç
robot en les variables \texttt{prx} i \texttt{pry}.

\begin{minted}[frame=lines, fontsize=\small]{vbnet}
2010 *CAM2ROB
2020 	angleR =(angle*M_PI) / 180.0
2030    prx = ppx*sin(angleR) + ppy*cos(angleR) + oPy + cPx
2040    pry = ppx*cos(angleR) - ppy*sin(angleR) + oPx + cPy
2050    return
\end{minted}

\paragraph{Obertura-Tancament de pinça}
Per tal de facilitar la llegibilitat del codi i no afegir constantment
els retards per la obertura i tancament de pinça s'han definit les funcions
\texttt{OPINCA} i \texttt{TPINCA} que encapsulen aquest comportament.

\begin{minted}[frame=lines, fontsize=\small]{vbnet}
2070 *OPINCA
2080    dly DOPINCAI
2090    hopen PINCA
2100    dly DOPINCAF
2110    return
\end{minted}
\begin{minted}[frame=lines, fontsize=\small]{vbnet}
2120 *TPINCA
2130    dly DTPINCAI
2140    hclose PINCA
2150    dly DTPINCAF
2160    return
\end{minted}

\subsubsection{Inicialitzacions}
En aquesta rutina es duu a terme l'operació de \emph{homing} i demés
inicialitzacions físiques del braç robot com l'obertura de la pinça
o de les variables que controlen l'entorn del robot, com el 
comptador de peces que s'han posat a cada palé.

\begin{minted}[frame=lines, fontsize=\small]{vbnet}
1040 *INIT
1050    servo ON
1060    ovrd VNORMAL
1070    mov Paralisi
1080    gosub *OPINCA
1090    for i = 1 to TP
1100        alloc(i) = 0
1110    next
1120    return
\end{minted}

\subsubsection{Càlcul de punts}\label{codcalcpts}
Com ja s'ha comentat en la pràctica s'ha fet el màxim esforç per posar tots els
punts en funció d'uns pocs, per això en la funció de càlcul de punts s'omplen
els vectors que contenen els punts homòlegs calculats a partir de les
\emph{macros} que descriuen l'entorn físic. En aquest punt també es calcula
l'ordre de recollida de les peces en funció de l'angle\ref{figrecpec}. Per limitacions del
llenguatge comentades a l'apartat de incidents TODO LABEL el codi no has
sortir gaire elegant.

Un cop acabada la funció càlcul de punts tots els vectors de punts estan apunt
per ser recorreguts per part del braç robot en els eixos (X,Y), el Z és
calculat en temps d'execució ja que depèn de quina peça s'estigui tractant,
ja que son apilades a l'eix Z.
Aquests vectors emprats de interfície són:

\begin{description}
 \item [\texttt{Pila}] Cada component indica la posició (X,Y) de cada una de
les piles d'on s'agafen
inicialment les peces, en l'execució per defecte seran dues.
 \item [\texttt{Pale}] Cada component correspon a la posició on anirà
co\lgem ocat cada tipus de peça al palé.
 \item [\texttt{PalI}] Posició inicial on es co\lgem coca el braç per iniciar
el descens sobre el pal.
 \item [\texttt{PalF}] Punt final de descens del braç, on es considera que la
peça ja hi ha entrat i pot ser amollada.
\end{description}

\label{casosrot}
\begin{minted}[frame=lines, fontsize=\small]{vbnet}
1130 *CALCPTS
1140    for pil = 1 to PILES
1150        Pila(pil) = Pila0
1160        Pila(pil).x = Pila(1).x + DXPILES * (pil - 1)
1170        Pila(pil).y = Pila(1).y + DYPILES * (pil - 1) 'es 0
1180    next

1200    for tipus = 1 to TP
1210        ppx = RELPALE(tipus, 1) 'paramentres de entrada de CAM2ROB
1220        ppy = RELPALE(tipus, 2)
1230        gosub *CAM2ROB           'prx i pry son parametres de sortida
1240        Pale(tipus) = Pale0    'Escriu altres components i orientacions
1250        Pale(tipus).x = prx
1260        Pale(tipus).y = pry
1270        PaleOut(tipus) = PaleOut0
1280        PaleOut(tipus).x = prx + 5.0
1290        PaleOut(tipus).y = pry -20.0
1300    next

1320    for pal = 1 to PALS
1330        PalI(pal) = Pal0
1340        PalI(pal).x = PalI(pal).x + DXPALS * (pal - 1)
1350        PalI(pal).y = PalI(pal).y + DYPALS * (pal - 1)
1360        PalF(pal) = PalI(pal)
1370        PalF(pal).z = PalF(pal).z - DZPAL
1380    next
1390 	
1400    angle = angle mod 360
1410 	if ((0.0 < angle) and (angle < 45.0)) then
1420        ordre(1) = 3
1430        ordre(2) = 2
1440        ordre(3) = 4
1450        ordre(4) = 1
1460        return
1470  endif
1480  if ((45.0 < angle) and (angle < 90.0)) then 
1490        ordre(1) = 3
1500        ordre(2) = 4
1510        ordre(3) = 2
1520        ordre(4) = 1
1530        return
1540  endif
1550  if ((90.0 < angle) and (angle < 135.0)) then
1560        ordre(1) = 4
1570        ordre(2) = 3
1580        ordre(3) = 1
1590        ordre(4) = 2
1600        return
1610  endif
1620  if ((135.0 <= angle) and (angle < 180.0)) then
1630        ordre(1) = 4
1640        ordre(2) = 1
1650        ordre(3) = 3
1660        ordre(4) = 2
1670 		return
1680 	endif
1690  if ((180.0 <= angle) and (angle < 225.0)) then 
1700        ordre(1) = 1
1710        ordre(2) = 4
1720        ordre(3) = 2
1730        ordre(4) = 3
1740        return
1750  endif
1760  if ((225.0 <= angle) and (angle < 270.0)) then
1770        ordre(1) = 1
1780        ordre(2) = 2
1790        ordre(3) = 4
1800        ordre(4) = 3
1810        return
1820  endif
1830  if ((270.0 <= angle) and (angle < 315.0)) then 
1840        ordre(1) = 2
1850        ordre(2) = 1
1860        ordre(3) = 4
1870        ordre(4) = 3
1880 	return
1890  endif
1900  ' 315.0 > angle < 360.0
1910  ordre(1) = 2
1920  ordre(2) = 3
1930  ordre(3) = 1
1940  ordre(4) = 4
1950  return
\end{minted}

\subsubsection{Munta palé}
Tal i com indica el títol en aquesta funció es munta el palé, aquesta tasca
consisteix en iterar per cada una de les peces existents en cada una de les
piles per agafar-les i depositar-les al lloc corresponent del palé.

\begin{minted}[frame=lines, fontsize=\small]{vbnet}
2170 *MNTPALE
2180    peca = 1
2190    for pil = 1 to PILES 
2200        for pecapila = 1 to npcspila(pil)
2210            gosub *AGAFPILA
2220            gosub *POSAPALE
2230            peca = peca + 1
2240        next
2250    next
2260    return
\end{minted}

\paragraph{Agafa de la pila}
\begin{minted}[frame=lines, fontsize=\small]{vbnet}
2370 *AGAFPILA
2380    mov Pila(pil)
2390    Prvsnl = Pila(pil)
2400    Prvsnl.z = Prvsnl.z - ZTPR + (npcspila!(pil) - pecapila) * HDISC
2410    ovrd VLENT
2420    mvs Prvsnl
2430    gosub *TPINCA
2440    mvs Pila(pil)
2450    ovrd VNORMAL
2460    return
\end{minted}

\paragraph{Posa al Palé}
\begin{minted}[frame=lines, fontsize=\small]{vbnet}
2470 *POSAPALE
2480    mov Pale(tipusP(peca))
2490    Prvsnl = Pale(tipusP(peca))
2500    Prvsnl.z = Prvsnl.z - ZTPR + alloc(tipusP(peca)) * HDISC
2510    ovrd VLENT
2520    mvs Prvsnl
2530    gosub *OPINCA
2540    mvs Pale(tipusP(peca))
2550    ovrd VNORMAL
2560    alloc(tipusP(peca)) = alloc(tipusP(peca)) + 1
2570    return
\end{minted}

\subsubsection{Desmunta palé}
En aquest punt del programa el braç va munt a munt del palé agafant les peces
i deixant-les als pals, no passa al següent munt mentre a l'actual quedin peces.
L'ordre d'agafada dels munts ha estat calculat en \emph{calcul de punts} com ja s'ha
esmentat\ref{codcalcpts}.

\paragraph{Agafa del palé}
\begin{minted}[frame=lines, fontsize=\small]{vbnet}
2580 *AGAFPALE
2590    mov PaleOut(munt)
2600    Prvsnl = PaleOut(munt)
2610    Prvsnl.z = Prvsnl.z - ZTPT + (alloc(munt) - 1) * HDISC
2620    ovrd VLENT
2630    mvs Prvsnl
2640    gosub *TPINCA
2650    mvs PaleOut(munt)
2660    ovrd VNORMAL
2670    alloc(munt) = alloc(munt) - 1
2680    return
\end{minted}


\paragraph{Posa pal}
\begin{minted}[frame=lines, fontsize=\small]{vbnet}
2690 *POSADEST
2700    mov PalI(munt)
2710    ovrd VLENT
2720    mvs PalF(munt)
2730    gosub *OPINCA
2740    mvs PalI(munt)
2750    ovrd VNORMAL
2760 	return
\end{minted}

\subsubsection{Acaba}
De manera similar a l'inicialitza, acaba s'encarrega del posicionament
físic del robot en una posició de repòs i la desconnexió dels servomotors.

\begin{minted}[frame=lines, fontsize=\small]{vbnet}
1960 *ACABA
1970    mov Paralisi
1980    hopen PINCA
1990    servo OFF
2000    return
\end{minted}

